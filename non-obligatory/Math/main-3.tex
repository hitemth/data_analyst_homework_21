\documentclass[a4paper, 12pt]{article}
\usepackage[utf8]{inputenc}
\usepackage[warn]{mathtext}
\usepackage[russian]{babel}
\usepackage[T2]{fontenc}
\usepackage[warn]{mathtext}
\usepackage{caption}

\usepackage{graphicx}
\graphicspath{ {images/} }
\usepackage{tikz}
\usepackage{pgfplots}

\usepackage{amsmath}
\usepackage{floatflt}
\usepackage[left=20mm, top=20mm, right=20mm, bottom=20mm, footskip=10mm]{geometry}

\usepackage{multicol}
\setlength{\columnsep}{2cm}


\usepackage{multicol}
\setlength{\columnsep}{2cm}
\usepackage{hyperref}

\author{Ивашкина Екатерина}
\begin{document}
\begin{titlepage}
	\centering
	\vspace{5cm}
	{\scshape . \par}
	\vspace{4cm}
	{\scshape\Large Решения математической части отбора \par}
	\vspace{1cm}
	{\Large\bfseries Ивашкина Екатерина \par}
	\vspace{1cm}
	\vfill

	\vfill

	

	\vfill

% Bottom of the page
        2021 г.
\end{titlepage}
\newpage
\paragraph{Задача 1}
В игре «Что? Где? Когда?» в каждом раунде волчок останавливается в секторе номер n, где n равновероятно принимает одно из значений 0, 1,..., 13. При этом играет первый из секторов по часовой стрелке, который ранее не играл. Найдите вероятность того, что после шести раундов сыграют (в любом порядке) секторы 1, 2,..., 6.
\paragraph{Решение}
Определим модель. Элементарным событием является один проведенный эксперимент или остановка волчка на секторе $n$. Так как вероятность всех элементарных событий одинаковая и равна $\frac{1}{14}$ то мы работаем с классической моделью. В классической модели вероятность искомого события $P(A) = \frac{|A|}{|\Omega|}$, где $\Omega$ - множество всех элементарных событий. В данном случае $|\Omega| = 14^6$ - количество всевозможных упорядоченных выборок по 6 из 14 с повторениями.
Чтобы найти $P(A)$ легче рассмотреть задачу от обратного. Допустим у нас $7$ секторов, тогда $|A|$ - событие когда $1$ не выпадает, а вероятность этого события $1/7$. Тогда $|A| = P(A) \cdot |\Omega|$ = $1/7 \cdot 7^6 = 7^5$. А вероятность нашего исходного события с 14 секторами равна $7^5/14^6 = 1/448$.\\
Ответ:$1/448$


\paragraph{Задача 2}
Аналитик рынка ценных бумаг оценивает среднюю доходность определенного вида акций. Случайная выборка из $16$ дней показала, что средняя доходность по акциям данного типа составляет $8\%$ с выборочным средним квадратическим отклонением в $4\%$. Предполагая, что доходность акции подчиняется нормальному закону распределения, определите $99\%$ -ый доверительный интервал для средней доходности интересующего аналитика вида акций.
\paragraph{Решение}
В данной задаче доходность акции это случайная величина, средняя доходность - выборочное среднее. Так как нам неизвестна дисперсия генеральной выборки и объем выборки мал ($n = 16 < 30$), то нужно использовать t-распределение Стьюдента. Границы доверительного интервала для средней доходности тогда будут задаваться формулой
$(\overline{x} - t_{\alpha, n - 1}\cdot \frac{s}{\sqrt{n}}; \overline{x} + t_{\alpha, n-1}\cdot \frac{s}{\sqrt{n}})$. Где $t_{\alpha, n-1}$ - квантиль распределения Стьюдента уровня $1-\frac{\alpha}{2}$ c $n-1$ степенью сввободы. Так  как уровень доверия $\beta = 1 - \alpha = 0.99$, то $\alpha = 0.01$. Следовательно, можно найти значения квантиля в таблице t-распределния, оно будет равно $3.29$.
Доверительным интервалом будет являться интервал 
$(0.08 - 3.29 \cdot \frac{0.04}{\sqrt{16}}; 0.08 + 3.29\cdot \frac{0.04}{\sqrt{16}}) = (0.08 - 0.0329; 0.08 + 0.0329) = (0.0471; 0,1129)$ или $(4,71\%; 11,29\%)$.
Ответ: Доверительный интервал для средней доходности акции $(4,71\%; 11,29\%)$.






\paragraph{Задача 3}
Мужчины и женщины по-разному оценивают положительные человеческие качества. Предложили мужчинам и женщинам на основе десятибалльной шкалы (10 баллов – это максимум) оценить важность следующих пяти качеств в представителях противоположного пола:
\\
\\
\begin{minipage}{0.5\textwidth}
    \begin{center}
    \begin{tabular}{|l|l|l|l|l|l|}
    \hline
    Качества & Ум & Доброта  & Красота & Юмор & Работоспособность\\ \hline
    Мужчины & 7   & 8 & 8          & 5 & 7      \\ 
    \hline
    Женщины & 10   & 5  & 3          & 8 & 10      \\ 
    \hline
    \end{tabular}
    \label{table::no_lenses}
    \end{center}
\end{minipage}
\\
Найдите тесноту связи между этими данными, рассматривая данные, как выборочные наблюдения случайных величин. Сделайте вывод о том, насколько близки или далеки мужчины и женщины в оценках  качеств партнеров.

\paragraph{Решение}
Теснота связи характеризуется с помощью коэффициента корреляции: \\
$r(X,Y) = \frac{cov(X,Y)}{\sigma_X \sigma_Y} = \frac{\overline{XY} - \overline{X}\cdot \overline{Y}}{\sqrt{\overline{X^2} - \overline{X}^2}\cdot \sqrt{\overline{Y^2} - \overline{Y}^2}}$. Где $\overline{X} = \frac{1}{n} \sum\limits_{i=1}^n X_i$, $\overline{Y} = \frac{1}{n} \sum\limits_{i=1}^n Y_i$ - выборочные средние.
Составим расчетную таблицу:
\begin{minipage}{0.5\textwidth}
    \begin{center}
    \begin{tabular}{|l|l|l|l|l|l|l|}
    \hline
     &  & &  & & & Выб. среднее\\ \hline
    X & 7   & 8 & 8          & 5 & 7 &   7  \\ 
    \hline
    Y & 10   & 5  & 3          & 8 & 10 &   7,2  \\ 
    \hline
    $X^2$ & 49 &	64 &	64 &	25	& 49 & 50,2 \\
    \hline
    $Y^2$ & 100 &	25 &	9 &	64 &	100	&	59,6 \\
    \hline
    $X\cdot Y$ & 70 &	40 &	24 &	40 &	70	&	48,8 \\
    \hline
    \end{tabular}
    \label{table::no_lenses}
    \end{center}
\end{minipage}
\\
Подставим значения в формулу: 
$r(X,Y) = \frac{48,8 - 50,4}{\sqrt{50,2 - 49} \sqrt{59,6 - 51,84}} = \frac{-1,6}{1,095 \cdot 2,786} = -0,524$.
По шкале Чеддока качественной оценки тесноты связи коэффициенту корелляции $-0,524$ сопоставляется категория тесноты -"Заметная". \\
Ответ: количественная оценка тесноты связи - $-0,524$, качественная - Заметная. А значит мужчины и женщины, ожидаемо, имеют общие приоритеты в качествах партнера, но все таки у каждого пола есть свои особенно выраженные предпочтения, например ум в мужчинах для женщин. Средняя категория тесноты связи "Заметная" хорошо описывает эту ситуацию.
\end{document}